\newcommand{\v}[1]{\mathbf{#1}}
\newcommand{\dv}[1]{\dot{\v{#1}}}
\newcommand{\ddv}[1]{\ddot{\v{#1}}}
\newcommand{\uv}[1]{\hat{\v{#1}}}

\begin{document}

\maketitle

\section{Orbit Mechanics Past Problems}

\subsection{Spring 2021}

\subsubsection{Problem Statement}

Assume a system of four centrobaric bodies that can all move in any spatial dimension.

\begin{enumerate}
    \item From first principles, derive the vector differential equation governing relative motion. It is not possible to solve the corresponding scalar equations of motion. Why not?
    \item Derive expressions for the 10 known integrals of motion associated with this vector differential equation for the 4-body system. What is the physical significance of each?
    \item The motion of the Moon relative to the Earth, and influenced by the Sun, is one of the most challenging problems in orbital mechanics. Given the results in (a) and (b), discuss why the first statement is true.
\end{enumerate}

\subsubsection{Solution}

The vector from the $i$th body to the $j$th body is given by:

\begin{align*}
    \v{r}_{ij} &= \v{r}_j - \v{r}_i \\
\end{align*}

Taking two derivatives with respect to time:

\begin{align*}
    \ddv{r}_{ij} &= \ddv{r}_j - \ddv{r}_i \\
\end{align*}

The acceleration of the $i$th body is given by:

\begin{align*}
    \ddv{r}_i &= - \sum_{k\neq i} \frac{G m_k}{\left| \v{r}_{ki} \right|^3} \v{r}_{ki} \\
\end{align*}

The acceleration of the $j$th body is given by:

\begin{align*}
    \ddv{r}_j &= -\sum_{k\neq j} \frac{G m_k}{\left| \v{r}_{kj} \right|^3} \v{r}_{kj} \\
\end{align*}

Such that the relative acceleration of the $i$th body with respect to the $j$th body is given by:

\begin{align*}
    \ddv{r}_{ij} &= -\sum_{k\neq j} \frac{G m_k}{\left| \v{r}_{kj} \right|^3} \v{r}_{kj} + \sum_{k\neq i} \frac{G m_k}{\left| \v{r}_{ki} \right|^3} \v{r}_{ki} \\
\end{align*}

We can pull out the $k=i$ in the first sum and $k=j$ in the second sum:

\begin{align*}
    \ddv{r}_{ij} &= -\frac{G m_i}{\left| \v{r}_{ji} \right|^3} \v{r}_{ji} + \frac{G m_j}{\left| \v{r}_{ij} \right|^3} \v{r}_{ij} -\sum_{k\neq j, k\neq i} \frac{G m_k}{\left| \v{r}_{kj} \right|^3} \v{r}_{kj} + \sum_{k\neq i, k\neq j} \frac{G m_k}{\left| \v{r}_{ki} \right|^3} \v{r}_{ki} \\
\end{align*}

Combining the first two terms and the last two terms:

\begin{align*}
    \ddv{r}_{ij} &= -\frac{G (m_i + m_j)}{\left| \v{r}_{ji} \right|^3} \v{r}_{ji} - \sum_{k\neq j, k\neq i} G m_k \left(\frac{\v{r}_{kj}}{{\left| \v{r}_{kj} \right|^3}} - \frac{\v{r}_{ki}}{{\left| \v{r}_{ki} \right|^3}}\right)
\end{align*}

Which in the case of the 4-body problem, we take the $i=1$ and $j=2$, $k \in [3,4]$:

\begin{align*}
    \ddv{r}_{12} &= -\frac{G (m_1 + m_2)}{\left| \v{r}_{21} \right|^3} \v{r}_{21} - \sum_{k=3}^4 G m_k \left(\frac{\v{r}_{k2}}{{\left| \v{r}_{k2} \right|^3}} - \frac{\v{r}_{k1}}{{\left| \v{r}_{k1} \right|^3}}\right)
\end{align*}

Deriving the 10 known integrals of motion begins by first noting that the sum of all forces on the system is zero:

\begin{equation}
    \sum_i F = \sum_i \sum_{j \neq i} \left( -\frac{G m_i m_j}{r_{ji}^3} \v{r}_{ji} \right)
\end{equation}

Because $F_{sys} = m_{sys} a_{sys} = \sum_i m_i \ddv{r}_i = 0$ for the system as a whole, we can state:

\begin{equation}
    \int \sum_i m_i \ddv{r}_i \: dt = m_{sys} v_{sys} = \sum_i m_i \dv{r}_i
\end{equation}

Integrating once more:

\begin{equation}
    \int \sum_i m_i \dv{r}_i \: dt = m_{sys} v_{sys} t + m_{sys} r_{sys}  = \sum_i m_i \v{r}_i
\end{equation}

These two constants of integration $\v{r}_{sys}$ and $\v{v}_{sys}$ are the first two integrals of motion (making up six scalar equations). Specifically, they are solved for by dividing by $m_{sys}$:

\begin{align*}
    \v{r}_{sys} &= \frac{1}{m_{sys}} \sum_i m_i \v{r}_i \\
    \v{v}_{sys} &= \frac{1}{m_{sys}} \sum_i m_i \dv{r}_i \\
\end{align*}

The next three integrals of motion are found by taking the summation of the angular momentum of the system. We must develop this by taking the sum of the torques on members of the system. First, we note that the torque on the $i$th body is given by:

\begin{align*}
    \v{\tau}_i &= \v{r}_i \times \v{F}_i \\
    &= \v{r}_i \times \sum_{j \neq i} \left( -\frac{G m_i m_j}{r_{ji}^3} \v{r}_{ji} \right) \\
\end{align*}

Such that the total torque on the system is:

\begin{align*}
    \sum_i \v{\tau}_i &= \sum_i \v{r}_i \times \sum_{j \neq i} \left( -\frac{G m_i m_j}{r_{ji}^3} \v{r}_{ji} \right) \\
    &= \sum_i \sum_{j \neq i} \v{r}_i \times \left( -\frac{G m_i m_j}{r_{ji}^3} \v{r}_{ji} \right) \\
\end{align*}

We then note that $\v{r}_i \times \v{r}_{ji} = -\v{r}_{ji} \times \v{r}_i$, such that each term in the summation is annihilated by its counterpart. This leaves us with:

\begin{align*}
    \sum_i \v{\tau}_i &= \v{0} \\
    &= \sum_i \v{r}_i \times m_i \ddv{r}_i
\end{align*}

We notice that this summation expansion of the torque is the derivative of another quantity:

\begin{align*}
    \sum_i \v{r}_i \times m_i \ddv{r}_i &= \frac{d}{dt} \left( \sum_i \v{r}_i \times m_i \dv{r}_i \right) \\
\end{align*}

Which implies that the integral of the derived quantity is an integral of motion:

\begin{align*}
    \sum_i \v{r}_i \times m_i \dv{r}_i &= \v{h}_{sys} \\
\end{align*}

Finally, we enforce conservation of energy by first finding the potential function whose gradient is the force on the system:

\begin{align*}
    U &= \frac{1}{2} G \sum_i \sum_{j \neq i} \frac{m_i m_j}{r_{ji}} \\
\end{align*}

Such that the gradient of the potential function is:

\begin{align*}
    \nabla U &= -G \sum_i \sum_{j \neq i} \frac{m_i m_j}{r_{ji}^3} \v{r}_{ji} \\
    &= \sum_i m_i \ddv{r} \\
\end{align*}

Notice that we can express:

\begin{align*}
    \sum_i \nabla U &= \sum_i\frac{d U}{d \v{r}_i} \\
    &= \sum_i m_i \ddv{r}
\end{align*}

If we multiply both sides by $\dv{r}$:

\begin{align*}
    \sum_i m_i \dv{r}_i \cdot \ddv{r}_i &= \sum_i\frac{d U}{d \v{r}_i} \frac{d \v{r}_i}{dt} \\
    &= \frac{d U}{dt} \\
\end{align*}

We then notice that the left side can also be expressed as the derivative of a quantity:

\begin{align*}
    \frac{dU}{dt} &= \frac{d}{dt} \left( \sum_i m_i \dot{r}_i^2 \right) \\
    U = \sum_i m_i \dot{r}_i^2 + C_2 \\
\end{align*}

Where $C_2$ is the total system energy. Plugging in our particle system representation of $U$, we find that the total system energy is given by:

\begin{align*}
    C_2 &= \sum_i m_i \dot{r}_i^2 - \frac{1}{2} G \sum_i \sum_{j \neq i} \frac{m_i m_j}{r_{ji}} \\
\end{align*}


\subsection{Fall 2023}

Note that this problem was also given in Fall 2021.

\subsubsection{Problem Statement}

Assuming Keplerian motion, several important types of orbital maneuvers are noncoplanar. For example, the capability to change the ascending node can widen the launch window.

Assume that the orbital elements for an Earth orbit are given.

\begin{enumerate}
    \item To change only the ascending node, derive an equation (or equations) that, if solved, will identify the location. i.e.\ the argument of latitude, for the location of the maneuver in the original and final orbits.
    \item If the orbit is circular, let $e=0.0$, $i=55^\circ$, $\Omega_i=0^\circ$, $\Omega_f=45^\circ$, where $o$ reflects the original orbit and $f$ indicates a value in the final orbit. In the relationships from (a), demonstrate that the maneuver location is defined as $\theta_o = 103.36^\circ$. What is the value of $\theta_f$?
    \item If the circular orbit possesses a radius of $3R_\oplus$, find the required $\Delta v$.
\end{enumerate}

\subsubsection{Solution}

We can form a spherical triangle with side lengths $\Omega_f - \Omega_o$ along the equator, and then $\theta_i$ extending upwards from the left at an angle of $i_o$, and $\theta_f$ extending upwards from the right at an angle of $180^\circ-i_o$. The angle at the top of the triangle is the angle between the initial and final position vectors, which is the angle of the required $\Delta v$. We can then use the spherical law of cosines to solve for this angle:

\begin{align*}
    \cos a &= \cos b \cos c + \sin b \sin c \cos A \\
    \cos A &= - \cos b \cos c + \sin b \sin c \cos a \\
\end{align*}

Where the lowercase letters are the side lengths and the uppercase letters are the interior angles. Rephrased for our problem, we find the third interior angle $a_3$:

\begin{align*}
    \cos a_3 &= - \cos i_o \cos (180^\circ - i_f) + \sin i_o \sin (180^\circ - i_f) \cos(\Omega_f - \Omega_o) \\
    &= \cos^2 55^\circ + \sin^2 55^\circ \cos(45^\circ) \\
    &= \cos^2 55^\circ + \frac{\sqrt{2}}{2} \sin^2 55^\circ \\
    a_3 &= \cos^{-1} \left( \cos^2 55^\circ + \frac{\sqrt{2}}{2} \sin^2 55^\circ \right) \\
    &\approx 37^\circ \\
\end{align*}

We can then use the spherical law of sines to solve for $\theta_o$:

\begin{align*}
    \frac{\sin\theta_o}{\sin i_f} &= \frac{\sin(\Omega_f - \Omega_o)}{\sin a_3} \\
    \sin\theta_o &= \frac{\sin i_f \sin(\Omega_f - \Omega_o)}{\sin a_3} \\
    \theta_o &= 76.64^\circ \\
\end{align*}

Notice that the arcsin is also solved by $\theta_o = 180^\circ - 76.64^\circ = 103.36^\circ$. We choose this solution to yield an intersection in the first half of the initial orbit.

Solving for $\theta_f$ similarly:

\begin{align*}
    \frac{\sin\theta_f}{\sin i_o} &= \frac{\sin(\Omega_f - \Omega_o)}{\sin a_3} \\
    \sin\theta_f &= \frac{\sin i_o \sin(\Omega_f - \Omega_o)}{\sin a_3} \\
    \theta_f &= 76.64^\circ \\
\end{align*}

We can then find the magnitude of the required $\Delta v$ using the law of cosines by recognizing that the magnitude of the velocity is the same for both the initial and final orbits:

\begin{align*}
    v_c &= \sqrt{\frac{\mu_\oplus}{r}} \\
    &= \sqrt{\frac{\mu_\oplus}{3R_\oplus}} \\
\end{align*}

And the magnitude of the required $\Delta v$ is given by:

\begin{align*}
    \frac{\Delta v}{2 v_c} &= \sin\left( \frac{a_3}{2} \right) \\
    &= \sin\left( \frac{37^\circ}{2} \right) \\
    &\approx 0.30 \\
    \Delta v &\approx 0.60 v_c \\
    &\approx 0.60 \sqrt{\frac{\mu_\oplus}{3R_\oplus}} \\
    &\approx 2.86 \: [km/s] \\
\end{align*}

This concludes the derivation of the ten integrals of motion for the n-body problem. The first six scalars are the initial position and velocity of the system, and the next three are the angular momentum of the system. The final scalar is the total energy of the system.

\subsection{Problem 0}

\subsubsection{Problem Statement}

In Keplerian mechanics, several important types of orbital maneuvers are noncoplanar. For example, the capability to change both the ascending node and the inclination with only one maneuver is efficient and can widen the launch window.

Assume that the orbital elements for an Earth orbit are given. If the orbit is circular both initially and after the maneuver, let $i_o=30^\circ$, $i_f=90^\circ$, $\Omega_o=0^\circ$, $\Omega_f=60^\circ$, where $o$ reflects the original orbit and $f$ indicates a value in the final orbit.

\begin{enumerate}
    \item Determine the appropriate maneuver location in each orbit.
    \item If the circular orbit possesses a radius of $4R_\oplus$, determine the magnitude of the required single impulse to accomplish the goal.
\end{enumerate}

\subsubsection{Solution}

We'll define the ``location'' of the maneuver in the initial and final orbits with the argument of latitude $\theta_o$ and $\theta_f$, the angle between the ascending node and the spacecraft's position vector. Because the orbits are circular, we can't really use the true anomaly. We can then form a spherical triangle with side lengths $\Omega_f - \Omega_o$ along the equator, and then $\theta_i$ extending upwards from the left at an angle of $i_o$, and $\theta_f$ extending upwards from the right at an angle of $i_f$. Note: that in general, a spherical triangle has a sum of interior angles greater than $180^\circ$. This means that we must solve for the interior angle at the top of the triangle using the spherical law of cosines:

\begin{align*}
    \cos a &= \cos b \cos c + \sin b \sin c \cos A \\
    \cos A &= - \cos b \cos c + \sin b \sin c \cos a \\
\end{align*}

Where the lowercase letters are the side lengths and the uppercase letters are the interior angles. Rephrased for our problem, we find the third interior angle $a_3$:

\begin{align*}
    \cos a_3 &= - \cos i_o \cos i_f + \sin i_o \sin i_f \cos(\Omega_f - \Omega_o) \\
    &= - \cos 30^\circ \cos 90^\circ + \sin 30^\circ \sin 90^\circ \cos(60^\circ - 0^\circ) \\
    &= - \frac{\sqrt{3}}{2} \cdot 0 + \frac{1}{2} \cdot 1 \cdot \frac{1}{2} \\
    &= \frac{1}{4} \\
    a_3 &= \cos^{-1} \left( \frac{1}{4} \right) \approx 76^\circ \\
\end{align*}

Using the spherical law of sines, we can solve for $\theta_o$:

\begin{align*}
    \frac{\sin\theta_o}{\sin i_f} &= \frac{\sin(\Omega_f - \Omega_o)}{\sin a_3} \\
    \sin\theta_o &= \frac{\sin i_f \sin(\Omega_f - \Omega_o)}{\sin a_3} \\
\end{align*}

Plugging in values, we find:

\begin{align*}
    \sin\theta_o &= \frac{\sin 90^\circ \sin(60^\circ)}{\sin 76^\circ} \\
    &= \frac{\sin 60^\circ}{\sin 76^\circ} \\
    &\approx 0.89 \\
    \theta_o &\approx 63^\circ
\end{align*}

And similarly for $\theta_f$:

\begin{align*}
    \frac{\sin\theta_f}{\sin i_o} &= \frac{\sin(\Omega_f - \Omega_o)}{\sin a_3} = 1 \\
    \sin\theta_f &= \sin 30^\circ \frac{\sin(60^\circ)}{\sin 76^\circ} \\
    &\approx 0.63 \\
    \theta_f &\approx 26.5^\circ
\end{align*}

The magnitude of the required impulse is given by the law of cosines, where we know that the angle between the initial and final position vectors is $a_3 \approx 76^\circ$, the interior angle of the spherical triangle at the point of intersection. The circular velocity in the initial orbit is given by:

\begin{align*}
    v_c &= \sqrt{\frac{\mu_\oplus}{r}} \\
    &= \sqrt{\frac{\mu_\oplus}{4R_\oplus}} \\
\end{align*}

And the magnitude of the required impulse is given by:

\begin{align*}
    \frac{\Delta v}{2 v_c} &= \sin\left( \frac{76^\circ}{2} \right) \\
    &\approx 0.62 \\
    \Delta v &\approx 1.23 v_c \\
    &\approx 1.23 \sqrt{\frac{\mu_\oplus}{4R_\oplus}} \\
\end{align*}

\subsection{Fall 2019}

\subsubsection{Problem Statement}
Consider a hyperbolic flyby of a planet

\begin{enumerate}
    \item Determine the values of the periapsis flyby radius $r_p$ and hyperbolic excess speed $v_\infty$ that yield the \textit{maximum possible} magnitude of the equivalent $\Delta v_{eq}$ for the spacecraft due to the flyby. Express your answer for $r_p$ in terms of the planet radius $r_s$; include the constraint that $r_p \geq r_s$.
    \item Determine this maximum $\Delta v_{eq}$ in terms of $v_s$, the circular speed at the surface of the planet. Also determine the numerical values for the corresponding turn angle $\delta$ and the hyperbolic eccentricity $e$.
\end{enumerate}

\subsubsection{Solution}

We know that the angle between the incoming and outgoing hyperbolic asymptotes is given by:

\begin{align*}
    \delta &= 2 \sin^{-1} \left( \frac{1}{e} \right) \\
    &= 2 \sin^{-1} \left( \frac{\Delta v_{eq}}{2 v_\infty} \right)
\end{align*}

We'll use these two expressions for $\delta$ to solve for the conditions that maximize $\Delta v$. First, we have to find a way to introduce $r_p$ into the equation. We know that the distance from the attracting focus to the center of the hyperbola is given by:

\begin{align*}
    ae &= r_p + a \\
    e &= \frac{r_p}{a} + 1
\end{align*}

We also know that by conservation of energy at $r=\infty$, we can express the semi-major axis $a$ in terms of the hyperbolic excess speed $v_\infty$:

\begin{align*}
    \frac{v_\infty^2}{2} &= \frac{\mu}{2a} \\
    a &= \frac{\mu}{v_\infty^2}
\end{align*}

Substituting this into the expression for $e$:

\begin{align*}
    e &= \frac{r_p}{\mu/v_\infty^2} + 1 \\
    &= \frac{r_p v_\infty^2}{\mu} + 1
\end{align*}

Such that we can equate the two expressions for $\delta$:

\begin{align*}
    2 \sin^{-1} \left( \frac{\Delta v_{eq}}{2 v_\infty} \right) &= 2 \sin^{-1} \left( \frac{1}{\frac{r_p v_\infty^2}{\mu} + 1} \right) \\
    \frac{\Delta v_{eq}}{2 v_\infty} &= \frac{1}{\frac{r_p v_\infty^2}{\mu} + 1} \\
    \Delta v_{eq} &= \frac{2 v_\infty}{\frac{r_p v_\infty^2}{\mu} + 1} \\
\end{align*}

This tells us that for any given $v_\infty$, minimizing $r_p$ will maximize $\Delta v_{eq}$. The minimum value of $r_p$ is $r_s$, the radius of the planet. Solving for the $v_\infty$ that corresponds to this minimum $r_p$ requires taking the derivative of the $\Delta v_{eq}$ expression with respect to $v_\infty$ and looking for critical points:

\begin{align*}
    \frac{\partial \Delta v_{eq}}{\partial v_\infty} &= \frac{2}{\frac{r_p v_\infty^2}{\mu} + 1} - \frac{2 v_\infty}{\left( \frac{r_p v_\infty^2}{\mu} + 1 \right)^2} \frac{2 r_p v_\infty}{\mu} \\
    &= \frac{\frac{2r_p v_\infty^2}{\mu} + 2 - \frac{4r_p v_\infty^2}{\mu}}{\left( \frac{r_p v_\infty^2}{\mu} + 1 \right)^2} \\
    &= \frac{2 - \frac{2r_p v_\infty^2}{\mu}}{\left( \frac{r_p v_\infty^2}{\mu} + 1 \right)^2} \\
\end{align*}

We notice that the denominator is always positive, so we can simply set the numerator to zero:

\begin{align*}
    2 - \frac{2r_p v_\infty^2}{\mu} &= 0 \\
    \frac{2r_p v_\infty^2}{\mu} &= 2 \\
    v_\infty^2 &= \frac{\mu}{r_p} \\
    v_\infty &= \sqrt{\frac{\mu}{r_p}}
\end{align*}

This is an interesting result! We have found that the hyperbolic excess velocity for maximum $\Delta v_{eq}$ is equal to the circular velocity at the surface of the planet. Solving for the corresponding value of $\Delta v_{eq}$:

\begin{align*}
    \Delta v_{eq} &= \frac{2 v_\infty}{\frac{r_p v_\infty^2}{\mu} + 1} \\
    &= \frac{2 \sqrt{\frac{\mu}{r_p}}}{\frac{r_p \left( \sqrt{\frac{\mu}{r_p}} \right)^2}{\mu} + 1} \\
    &= \frac{2 \sqrt{\frac{\mu}{r_p}}}{\frac{\mu}{\mu} + 1} \\
    &= \frac{2 \sqrt{\frac{\mu}{r_p}}}{2} \\
    &= \sqrt{\frac{\mu}{r_p}}
\end{align*}

We can also solve for the corresponding values of $\delta$:

\begin{align*}
    \delta &= 2 \sin^{-1} \left( \frac{1}{e} \right) \\
    &= 2 \sin^{-1} \left( \frac{\Delta v_{eq}}{2 v_\infty} \right) \\
    &= 2 \sin^{-1} \left( \frac{\sqrt{\frac{\mu}{r_p}}}{2 \sqrt{\frac{\mu}{r_p}}} \right) \\
    &= 2 \sin^{-1} \left( \frac{1}{2} \right) \\
    &= 60^\circ \\
\end{align*}

And $e$:

\begin{align*}
    e &= \frac{r_p}{a} + 1 \\
    &= \frac{r_p}{\frac{\mu}{v_\infty^2}} + 1 \\
    &= \frac{r_p v_\infty^2}{\mu} + 1 \\
    &= \frac{\mu}{\mu} + 1 \\
    &= 2
\end{align*}

\end{document}
