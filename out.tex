% Options for packages loaded elsewhere
\PassOptionsToPackage{unicode}{hyperref}
\PassOptionsToPackage{hyphens}{url}
%
\documentclass[
]{article}

\let\oldsection\section
\renewcommand\section{\clearpage\oldsection}


\usepackage{amsmath,amssymb}
\usepackage{iftex}
\ifPDFTeX
  \usepackage[T1]{fontenc}
  \usepackage[utf8]{inputenc}
  \usepackage{textcomp} % provide euro and other symbols
\else % if luatex or xetex
  \usepackage{unicode-math} % this also loads fontspec
  \defaultfontfeatures{Scale=MatchLowercase}
  \defaultfontfeatures[\rmfamily]{Ligatures=TeX,Scale=1}
\fi
\usepackage{lmodern}
\ifPDFTeX\else
  % xetex/luatex font selection
\fi
% Use upquote if available, for straight quotes in verbatim environments
\IfFileExists{upquote.sty}{\usepackage{upquote}}{}
\IfFileExists{microtype.sty}{% use microtype if available
  \usepackage[]{microtype}
  \UseMicrotypeSet[protrusion]{basicmath} % disable protrusion for tt fonts
}{}
\makeatletter
\@ifundefined{KOMAClassName}{% if non-KOMA class
  \IfFileExists{parskip.sty}{%
    \usepackage{parskip}
  }{% else
    \setlength{\parindent}{0pt}
    \setlength{\parskip}{6pt plus 2pt minus 1pt}}
}{% if KOMA class
  \KOMAoptions{parskip=half}}
\makeatother
\usepackage[margin=0.5in]{geometry}
\usepackage{xcolor}
\setlength{\emergencystretch}{3em} % prevent overfull lines
\providecommand{\tightlist}{%
  \setlength{\itemsep}{0pt}\setlength{\parskip}{0pt}}
\setcounter{secnumdepth}{5}
\ifLuaTeX
  \usepackage{selnolig}  % disable illegal ligatures
\fi
\IfFileExists{bookmark.sty}{\usepackage{bookmark}}{\usepackage{hyperref}}
\IfFileExists{xurl.sty}{\usepackage{xurl}}{} % add URL line breaks if available
\urlstyle{same}
\hypersetup{
  hidelinks,
  pdfcreator={LaTeX via pandoc}}

\author{}
\date{}

\begin{document}

\section{Astrodynamics and Space Applications Qualifying Exam
Notes}\label{astrodynamics-and-space-applications-qualifying-exam-notes}

gps

orb

quals

sad

stm

\section{Satellite Navigation Past
Problems}\label{satellite-navigation-past-problems}

\section{Orbit Mechanics Past
Problems}\label{orbit-mechanics-past-problems}

\subsection{Problem 0}\label{problem-0}

\subsubsection{Problem statement}\label{problem-statement}

In Keplerian mechanics, several important types of orbital maneuvers are
noncoplanar. For example, the capability to change both the ascending
node and the inclination with only one maneuver is efficient and can
widen the launch window.

Assume that the orbital elements for an Earth orbit are given. If the
orbit is circular both initially and after the maneuver, let
\(i_0=30^\circ\), \(i_f=90^\circ\), \(\Omega_o=0^\circ\),
\(\Omega_f=60^\circ\), where \(o\) reflects the original orbit and \(f\)
indicates a value in the final orbit.

\begin{enumerate}
\tightlist
\item
  Determine the appropriate maneuver location in each orbit.
\item
  If the circular orbit possesses a radius of \(4R_\oplus\), determine
  the magnitude of the required single impulse to accomplish the goal.
\end{enumerate}

\subsubsection{Solution}\label{solution}

We'll define the ``location'' of the maneuver in the initial and final
orbits with the argument of latitude \(\theta_o\) and \(\theta_f\), the
angle between the ascending node and the spacecraft's position vector.
Because the orbits are circular, we can't really use the true anomaly.
We can then form a spherical triangle with side lengths
\(\Omega_f - \Omega_o\) along the equator, and then \(\theta_i\)
extending upwards from the left at an angle of \(i_0\), and \(\theta_f\)
extending upwards from the right at an angle of \(i_f\).

Using the spherical law of sines, we can solve for \(\theta_o\):

\[\begin{aligned}
\begin{aligned}
    \frac{\sin\theta_o}{\sin i_f} &= \frac{\sin(\Omega_f - \Omega_o)}{\sin\left(180^\circ-i_0-i_f\right)} \\
    \sin\theta_o &= \frac{\sin i_f \sin(\Omega_f - \Omega_o)}{\sin\left(180^\circ-i_0-i_f\right)} \\
\end{aligned}
\end{aligned}\]

Plugging in values, we find:

\[\begin{aligned}
\begin{aligned}
    \sin\theta_o &= \frac{\sin 90^\circ \sin(60^\circ - 0^\circ)}{\sin\left(180^\circ-30^\circ-90^\circ\right)} \\
    &= \frac{\sin 60^\circ}{\sin 60^\circ} \\
    &= 1 \\
    \theta_o &= 90^\circ
\end{aligned}
\end{aligned}\]

And similarly for \(\theta_f\):

\[\begin{aligned}
\begin{aligned}
    \frac{\sin\theta_f}{\sin i_o} &= \frac{\sin(\Omega_f - \Omega_o)}{\sin\left(180^\circ-i_0-i_f\right)} = 1 \\
    \theta_f &= i_o = 30^\circ
\end{aligned}
\end{aligned}\]

The magnitude of the required impulse is given by the law of cosines,
where we know that the angle between the initial and final position
vectors is \(i_f - i_o = 60^\circ\), the interior angle of the spherical
triangle at the point of intersection. The circular velocity in the
initial orbit is given by:

\[\begin{aligned}
\begin{aligned}
    v_c &= \sqrt{\frac{\mu_\oplus}{r}} \\
    &= \sqrt{\frac{\mu_\oplus}{4R_\oplus}} \\
\end{aligned}
\end{aligned}\]

And the magnitude of the required impulse is given by:

\[\begin{aligned}
\begin{aligned}
    \frac{\Delta v}{2 v_c} &= \sin\left( \frac{60^\circ}{2} \right) \\
    &= \frac{1}{2} \\
    \Delta v &= v_c \\
    &= \sqrt{\frac{\mu_\oplus}{4R_\oplus}} \\
\end{aligned}
\end{aligned}\]

\subsection{Problem 1}\label{problem-1}

\subsubsection{Problem statement}\label{problem-statement-1}

Consider a hyperbolic flyby of a planet

\begin{enumerate}
\tightlist
\item
  Determine the values of the periapsis flyby radius \(r_p\) and
  hyperbolic excess speed \(v_\infty\) that yield the \emph{maximum
  possible} magnitude of the equivalent \(\Delta v_{eq}\) for the
  spacecraft due to the flyby. Express your answer for \(r_p\) in terms
  of the planet radius \(r_s\); include the constraint that
  \(r_p \geq r_s\).
\item
  Determine this maximum \(\Delta v_{eq}\) in terms of \(v_s\), the
  circular speed at the surface of the planet. Also determine the
  numerical values for the corresponding turn angle \(\delta\) and the
  hyperbolic eccentricity \(e\).
\end{enumerate}

\subsubsection{Solution}\label{solution-1}

We know that the angle between the incoming and outgoing hyperbolic
asymptotes is given by:

\[\begin{aligned}
\begin{aligned}
    \delta &= 2 \sin^{-1} \left( \frac{1}{e} \right) \\
    &= 2 \sin^{-1} \left( \frac{\Delta v_{eq}}{2 v_\infty} \right)
\end{aligned}
\end{aligned}\]

We'll use these two expressions for \(\delta\) to solve for the
conditions that maximize \(\Delta v\). First, we have to find a way to
introduce \(r_p\) into the equation. We know that the distance from the
attracting focus to the center of the hyperbola is given by:

\[\begin{aligned}
\begin{aligned}
    ae &= r_p + a \\
    e &= \frac{r_p}{a} + 1
\end{aligned}
\end{aligned}\]

We also know that by conservation of energy at \(r=\infty\), we can
express the semi-major axis \(a\) in terms of the hyperbolic excess
speed \(v_\infty\):

\[\begin{aligned}
\begin{aligned}
    \frac{v_\infty^2}{2} &= \frac{\mu}{2a} \\
    a &= \frac{\mu}{v_\infty^2}
\end{aligned}
\end{aligned}\]

Substituting this into the expression for \(e\):

\[\begin{aligned}
\begin{aligned}
    e &= \frac{r_p}{\mu/v_\infty^2} + 1 \\
    &= \frac{r_p v_\infty^2}{\mu} + 1
\end{aligned}
\end{aligned}\]

Such that we can equate the two expressions for \(\delta\):

\[\begin{aligned}
\begin{aligned}
    2 \sin^{-1} \left( \frac{\Delta v_{eq}}{2 v_\infty} \right) &= 2 \sin^{-1} \left( \frac{1}{\frac{r_p v_\infty^2}{\mu} + 1} \right) \\
    \frac{\Delta v_{eq}}{2 v_\infty} &= \frac{1}{\frac{r_p v_\infty^2}{\mu} + 1} \\
    \Delta v_{eq} &= \frac{2 v_\infty}{\frac{r_p v_\infty^2}{\mu} + 1} \\
\end{aligned}
\end{aligned}\]

This tells us that for any given \(v_\infty\), minimizing \(r_p\) will
maximize \(\Delta v_{eq}\). The minimum value of \(r_p\) is \(r_s\), the
radius of the planet. Solving for the \(v_\infty\) that corresponds to
this minimum \(r_p\) requires taking the derivative of the
\(\Delta v_{eq}\) expression with respect to \(v_\infty\) and looking
for critical points:

\[\begin{aligned}
\begin{aligned}
    \frac{\partial \Delta v_{eq}}{\partial v_\infty} &= \frac{2}{\frac{r_p v_\infty^2}{\mu} + 1} - \frac{2 v_\infty}{\left( \frac{r_p v_\infty^2}{\mu} + 1 \right)^2} \frac{2 r_p v_\infty}{\mu} \\
    &= \frac{\frac{2r_p v_\infty^2}{\mu} + 2 - \frac{4r_p v_\infty^2}{\mu}}{\left( \frac{r_p v_\infty^2}{\mu} + 1 \right)^2} \\
    &= \frac{2 - \frac{2r_p v_\infty^2}{\mu}}{\left( \frac{r_p v_\infty^2}{\mu} + 1 \right)^2} \\
\end{aligned}
\end{aligned}\]

We notice that the denominator is always positive, so we can simply set
the numerator to zero:

\[\begin{aligned}
\begin{aligned}
    2 - \frac{2r_p v_\infty^2}{\mu} &= 0 \\
    \frac{2r_p v_\infty^2}{\mu} &= 2 \\
    v_\infty^2 &= \frac{\mu}{r_p} \\
    v_\infty &= \sqrt{\frac{\mu}{r_p}}
\end{aligned}
\end{aligned}\]

This is an interesting result! We have found that the hyperbolic excess
velocity for maximum \(\Delta v_{eq}\) is equal to the circular velocity
at the surface of the planet. Solving for the corresponding value of
\(\Delta v_{eq}\):

\[\begin{aligned}
\begin{aligned}
    \Delta v_{eq} &= \frac{2 v_\infty}{\frac{r_p v_\infty^2}{\mu} + 1} \\
    &= \frac{2 \sqrt{\frac{\mu}{r_p}}}{\frac{r_p \left( \sqrt{\frac{\mu}{r_p}} \right)^2}{\mu} + 1} \\
    &= \frac{2 \sqrt{\frac{\mu}{r_p}}}{\frac{\mu}{\mu} + 1} \\
    &= \frac{2 \sqrt{\frac{\mu}{r_p}}}{2} \\
    &= \sqrt{\frac{\mu}{r_p}}
\end{aligned}
\end{aligned}\]

We can also solve for the corresponding values of \(\delta\):

\[\begin{aligned}
\begin{aligned}
    \delta &= 2 \sin^{-1} \left( \frac{1}{e} \right) \\
    &= 2 \sin^{-1} \left( \frac{\Delta v_{eq}}{2 v_\infty} \right) \\
    &= 2 \sin^{-1} \left( \frac{\sqrt{\frac{\mu}{r_p}}}{2 \sqrt{\frac{\mu}{r_p}}} \right) \\
    &= 2 \sin^{-1} \left( \frac{1}{2} \right) \\
    &= 60^\circ \\
\end{aligned}
\end{aligned}\]

And \(e\):

\[\begin{aligned}
\begin{aligned}
    e &= \frac{r_p}{a} + 1 \\
    &= \frac{r_p}{\frac{\mu}{v_\infty^2}} + 1 \\
    &= \frac{r_p v_\infty^2}{\mu} + 1 \\
    &= \frac{\mu}{\mu} + 1 \\
    &= 2
\end{aligned}
\end{aligned}\]

\section{Attitude Dynamics Past
Problems}\label{attitude-dynamics-past-problems}

\section{Orbit Determination Past
Problem}\label{orbit-determination-past-problem}

\end{document}
